\documentclass[../homework.tex]{subfiles}
 \graphicspath{{\subfix{../images/}}}


\begin{document}

\newpage

\section{Приложение}

Приложение содержит исходный код компьютерной программы решения домашнего задания.
Программа написана на языке программирования Python~3.

\footnotesize

% Свой код помещаете внутри \begin{lstlisting}...\end{lstlisting}.
% Убирайте лишнии пустые строки, комментарии и другую бесполезность.
% Между импортами и кодом 2 пустые строки, между функциями и между функциями и основным кодом тоже 2 пустые строки.
% Внутри функций или основного кода допускается одна пустая строка, если это улучшает читаемость кода.
\begin{lstlisting}[
    language=Python,
    breaklines=true,
    numbers=left,
    numberstyle=\tiny
]
import numpy as np


def some_function(a, b):
    c = np.sqrt(a*b + 1)
    return c/a + b


result = some_function(4, 7)
with open("some_file.txt", "w") as f:
    f.write(f"Some result: {result}")

\end{lstlisting}

\end{document}