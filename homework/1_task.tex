% Преамбула отдельного файла общего документа
% (менять не стоит)
\documentclass[../homework.tex]{subfiles}
 \graphicspath{{\subfix{../images/}}}

\begin{document}

% Заголовок раздела (менять не стоит)
\section{Задание}

Для заданных параметров газа в камере (полное давление $\rho_{0}$, полная температура $T_{0}$, индивидуальная газовая постоянная $R_{0}$, показатель адиабаты $k$) и геометрии сопла (диаметр камеры $D_{к}$, диаметр критического сечения $d_{*}$, углы $\alpha$ и $\beta$) решить следующие задачи.

% Нумерованный список
\begin{enumerate}
\item
    По заданным геометрическим параметрам построить геометрию сопла Лаваля и график зависимости площади поперечного сечения сопла от координаты.
\item
    Определить газодинамические характеристики течения в сопле.
    \begin{enumerate}
    \item
        Найти распределение числа Маха и скоростной коэффициент $\lambda$ по координатам с использованием численных методов для решения нелинейных уравнений.
        Для этого как в дозвуковой, так и сверхзвуковой части нужно взять не менее 50 расчетных точек.
    \item
        Найти распределение всех газодинамических параметров (скорости течения газа, давления, температуры, плотности) по длине сопла и построить соответствующие графики.
        Проанализировать и объяснить изменение всех газодинамических величин по длине сопла.
    \item
        Проанализировать и объяснить поведение расхода, приходящегося на единицу площади поперечного сечения, по длине сопла.
    \item
        Построить график массового расхода $G$ по длине сопла и сделать соответствующие выводы.
    \end{enumerate}
\item
    Рассчитать силовые характеристики сопла.
    \begin{enumerate}
    \item
        При внешнем давлении равном атмосферному ($\rho_{н}$ = $10^{5}$~Па), вычислить тягу и удельную тягу.
    \item
        Определить степень нерасчетности сопла при заданных геометрических параметрах.
        Рассчитать потребную длину сопла, при которой оно работает на расчетном режиме.
    \item
        Рассчитать тягу сопла в вакууме и удельную тягу в вакууме.
        Определить идеальную скорость ракеты по формуле Циолковского при заданном отношении массы топлива к массе ракеты.
    \end{enumerate}
    \item
        Сформулировать соответствующие выводы.
\end{enumerate}

% Используйте ссылки вместо явного указания номеров рисунков, таблиц и пр.
В таблице~\ref{table:initial_data} приведены исходные данные для расчета.
В коде ссылка на таблицу выглядит так:
{\small
\begin{verbatim}
В таблице~\ref{table:initial_data}
приведены исходные данные для расчета.
\end{verbatim}
}

\begin{table}[h]
% Название таблицы
\caption{Исходные данные}
% Метка для ссылки
\label{table:initial_data}
% Выравнивание таблицы
\centering
% Перерасчёт ширины таблицы
\resizebox{\textwidth}{!}
{
    % Сама таблица
    \begin{tabular}{|c|c|c|c|c|c|c|c|c|c|}
        \hline
        $\rho_{0}$, МПа &
        $T_{0}$, К &
        $R$, Дж/(кг$\cdot$К) &
        $k$ &
        $d_{*}$, см &
        $\nu_{в}$, м/с &
        $D_{к}$ &
        $\alpha$ &
        $\beta$ &
        $\mu_т$ \\[3mm]
        \hline
        5.5 &
        2557 &
        339 &
        1.27 &
        53.1 &
        8.2 &
        1.9$d_{*}$ &
        43.9\degree &
        11.6\degree &
        0.67 \\
        \hline
    \end{tabular}
}
\end{table}

Вот как эта таблица выглядит в исходном файле:
{\small
\begin{verbatim}
\begin{table}[h]
    % Название таблицы
    \caption{Исходные данные}
    % Метка для ссылки
    \label{table:initial_data}
    % Выравнивание таблицы
    \centering
    % Перерасчёт ширины таблицы
    \resizebox{\textwidth}{!}
    {
        % Сама таблица
        \begin{tabular}{|c|c|c|c|c|c|c|c|c|c|}
            \hline
            $\rho_{0}$, МПа &
            $T_{0}$, К &
            $R$, Дж/(кг$\cdot$К) &
            $k$ &
            $d_{*}$, см &
            $\nu_{в}$, м/с &
            $D_{к}$ &
            $\alpha$ &
            $\beta$ &
            $\mu_т$ \\[3mm]
            \hline
            5.5 &
            2557 &
            339 &
            1.27 &
            53.1 &
            8.2 &
            1.9$d_{*}$ &
            43.9\degree &
            11.6\degree &
            0.67 \\
            \hline
        \end{tabular}
    }
\end{table}
\end{verbatim}
}

\end{document}
